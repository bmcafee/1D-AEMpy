% Options for packages loaded elsewhere
\PassOptionsToPackage{unicode}{hyperref}
\PassOptionsToPackage{hyphens}{url}
\PassOptionsToPackage{dvipsnames,svgnames,x11names}{xcolor}
%
\documentclass[
  letterpaper,
  DIV=11,
  numbers=noendperiod]{scrartcl}

\usepackage{amsmath,amssymb}
\usepackage{iftex}
\ifPDFTeX
  \usepackage[T1]{fontenc}
  \usepackage[utf8]{inputenc}
  \usepackage{textcomp} % provide euro and other symbols
\else % if luatex or xetex
  \usepackage{unicode-math}
  \defaultfontfeatures{Scale=MatchLowercase}
  \defaultfontfeatures[\rmfamily]{Ligatures=TeX,Scale=1}
\fi
\usepackage{lmodern}
\ifPDFTeX\else  
    % xetex/luatex font selection
\fi
% Use upquote if available, for straight quotes in verbatim environments
\IfFileExists{upquote.sty}{\usepackage{upquote}}{}
\IfFileExists{microtype.sty}{% use microtype if available
  \usepackage[]{microtype}
  \UseMicrotypeSet[protrusion]{basicmath} % disable protrusion for tt fonts
}{}
\makeatletter
\@ifundefined{KOMAClassName}{% if non-KOMA class
  \IfFileExists{parskip.sty}{%
    \usepackage{parskip}
  }{% else
    \setlength{\parindent}{0pt}
    \setlength{\parskip}{6pt plus 2pt minus 1pt}}
}{% if KOMA class
  \KOMAoptions{parskip=half}}
\makeatother
\usepackage{xcolor}
\setlength{\emergencystretch}{3em} % prevent overfull lines
\setcounter{secnumdepth}{-\maxdimen} % remove section numbering
% Make \paragraph and \subparagraph free-standing
\ifx\paragraph\undefined\else
  \let\oldparagraph\paragraph
  \renewcommand{\paragraph}[1]{\oldparagraph{#1}\mbox{}}
\fi
\ifx\subparagraph\undefined\else
  \let\oldsubparagraph\subparagraph
  \renewcommand{\subparagraph}[1]{\oldsubparagraph{#1}\mbox{}}
\fi


\providecommand{\tightlist}{%
  \setlength{\itemsep}{0pt}\setlength{\parskip}{0pt}}\usepackage{longtable,booktabs,array}
\usepackage{calc} % for calculating minipage widths
% Correct order of tables after \paragraph or \subparagraph
\usepackage{etoolbox}
\makeatletter
\patchcmd\longtable{\par}{\if@noskipsec\mbox{}\fi\par}{}{}
\makeatother
% Allow footnotes in longtable head/foot
\IfFileExists{footnotehyper.sty}{\usepackage{footnotehyper}}{\usepackage{footnote}}
\makesavenoteenv{longtable}
\usepackage{graphicx}
\makeatletter
\def\maxwidth{\ifdim\Gin@nat@width>\linewidth\linewidth\else\Gin@nat@width\fi}
\def\maxheight{\ifdim\Gin@nat@height>\textheight\textheight\else\Gin@nat@height\fi}
\makeatother
% Scale images if necessary, so that they will not overflow the page
% margins by default, and it is still possible to overwrite the defaults
% using explicit options in \includegraphics[width, height, ...]{}
\setkeys{Gin}{width=\maxwidth,height=\maxheight,keepaspectratio}
% Set default figure placement to htbp
\makeatletter
\def\fps@figure{htbp}
\makeatother
\newlength{\cslhangindent}
\setlength{\cslhangindent}{1.5em}
\newlength{\csllabelwidth}
\setlength{\csllabelwidth}{3em}
\newlength{\cslentryspacingunit} % times entry-spacing
\setlength{\cslentryspacingunit}{\parskip}
\newenvironment{CSLReferences}[2] % #1 hanging-ident, #2 entry spacing
 {% don't indent paragraphs
  \setlength{\parindent}{0pt}
  % turn on hanging indent if param 1 is 1
  \ifodd #1
  \let\oldpar\par
  \def\par{\hangindent=\cslhangindent\oldpar}
  \fi
  % set entry spacing
  \setlength{\parskip}{#2\cslentryspacingunit}
 }%
 {}
\usepackage{calc}
\newcommand{\CSLBlock}[1]{#1\hfill\break}
\newcommand{\CSLLeftMargin}[1]{\parbox[t]{\csllabelwidth}{#1}}
\newcommand{\CSLRightInline}[1]{\parbox[t]{\linewidth - \csllabelwidth}{#1}\break}
\newcommand{\CSLIndent}[1]{\hspace{\cslhangindent}#1}

\KOMAoption{captions}{tableheading}
\makeatletter
\@ifpackageloaded{tcolorbox}{}{\usepackage[skins,breakable]{tcolorbox}}
\@ifpackageloaded{fontawesome5}{}{\usepackage{fontawesome5}}
\definecolor{quarto-callout-color}{HTML}{909090}
\definecolor{quarto-callout-note-color}{HTML}{0758E5}
\definecolor{quarto-callout-important-color}{HTML}{CC1914}
\definecolor{quarto-callout-warning-color}{HTML}{EB9113}
\definecolor{quarto-callout-tip-color}{HTML}{00A047}
\definecolor{quarto-callout-caution-color}{HTML}{FC5300}
\definecolor{quarto-callout-color-frame}{HTML}{acacac}
\definecolor{quarto-callout-note-color-frame}{HTML}{4582ec}
\definecolor{quarto-callout-important-color-frame}{HTML}{d9534f}
\definecolor{quarto-callout-warning-color-frame}{HTML}{f0ad4e}
\definecolor{quarto-callout-tip-color-frame}{HTML}{02b875}
\definecolor{quarto-callout-caution-color-frame}{HTML}{fd7e14}
\makeatother
\makeatletter
\makeatother
\makeatletter
\makeatother
\makeatletter
\@ifpackageloaded{caption}{}{\usepackage{caption}}
\AtBeginDocument{%
\ifdefined\contentsname
  \renewcommand*\contentsname{Table of contents}
\else
  \newcommand\contentsname{Table of contents}
\fi
\ifdefined\listfigurename
  \renewcommand*\listfigurename{List of Figures}
\else
  \newcommand\listfigurename{List of Figures}
\fi
\ifdefined\listtablename
  \renewcommand*\listtablename{List of Tables}
\else
  \newcommand\listtablename{List of Tables}
\fi
\ifdefined\figurename
  \renewcommand*\figurename{Figure}
\else
  \newcommand\figurename{Figure}
\fi
\ifdefined\tablename
  \renewcommand*\tablename{Table}
\else
  \newcommand\tablename{Table}
\fi
}
\@ifpackageloaded{float}{}{\usepackage{float}}
\floatstyle{ruled}
\@ifundefined{c@chapter}{\newfloat{codelisting}{h}{lop}}{\newfloat{codelisting}{h}{lop}[chapter]}
\floatname{codelisting}{Listing}
\newcommand*\listoflistings{\listof{codelisting}{List of Listings}}
\makeatother
\makeatletter
\@ifpackageloaded{caption}{}{\usepackage{caption}}
\@ifpackageloaded{subcaption}{}{\usepackage{subcaption}}
\makeatother
\makeatletter
\@ifpackageloaded{tcolorbox}{}{\usepackage[skins,breakable]{tcolorbox}}
\makeatother
\makeatletter
\@ifundefined{shadecolor}{\definecolor{shadecolor}{rgb}{.97, .97, .97}}
\makeatother
\makeatletter
\makeatother
\makeatletter
\makeatother
\ifLuaTeX
  \usepackage{selnolig}  % disable illegal ligatures
\fi
\IfFileExists{bookmark.sty}{\usepackage{bookmark}}{\usepackage{hyperref}}
\IfFileExists{xurl.sty}{\usepackage{xurl}}{} % add URL line breaks if available
\urlstyle{same} % disable monospaced font for URLs
\hypersetup{
  pdftitle={1D-AEMpy v0.1a - Technical Documentation},
  pdfauthor={Robert Ladwig},
  pdfkeywords={keyword 1 aquatic ecosystem modeling},
  colorlinks=true,
  linkcolor={blue},
  filecolor={Maroon},
  citecolor={Blue},
  urlcolor={Blue},
  pdfcreator={LaTeX via pandoc}}

\title{1D-AEMpy v0.1a - Technical Documentation}
\author{Robert Ladwig}
\date{2024-01-09}

\begin{document}
\maketitle
\begin{abstract}
Technical documentation of the numerical implementation of the
one-dimensinal Aquatic Ecosystem Model in Pyton (1D-AEMpy). This
documentation describes the different modules that govern hydrodynamic
and water quality calculations to simulate the vertical dynamics of
water temperature, dissolved oxygen, phytoplankton biomass, nutrient
concentration, and aspects of organic carbon.
\end{abstract}
\ifdefined\Shaded\renewenvironment{Shaded}{\begin{tcolorbox}[breakable, frame hidden, sharp corners, interior hidden, enhanced, borderline west={3pt}{0pt}{shadecolor}, boxrule=0pt]}{\end{tcolorbox}}\fi

\hypertarget{overview}{%
\subsection{Overview}\label{overview}}

This documentation describes the current version 0.1a of 1D-AEMpy, which
includes hydrodynamic calculations coupled to a functional aquatic
ecosystem model that explicitly simulates dissolved oxygen,
phytoplankton biomass, a nutrient concentration (simulating reactive
phosphorus), labile particulate organic carbon (POC-l), labile dissolved
organic carbon (DOC-l), refractory particulate organic carbon (POC-r),
and refractory dissolved organic carbon (DOC-r). The coupling between
both models depends primarily on the light extinction coefficient
(function of the sum of organic carbon concentrations) to the heat
attenuation of the water column, as well as the derived eddy diffusivity
governing the vertical diffusion of dissolved water quality variables.

\begin{tcolorbox}[enhanced jigsaw, breakable, toprule=.15mm, rightrule=.15mm, bottomtitle=1mm, left=2mm, bottomrule=.15mm, leftrule=.75mm, opacityback=0, toptitle=1mm, arc=.35mm, coltitle=black, colframe=quarto-callout-note-color-frame, title=\textcolor{quarto-callout-note-color}{\faInfo}\hspace{0.5em}{Note}, opacitybacktitle=0.6, titlerule=0mm, colback=white, colbacktitle=quarto-callout-note-color!10!white]

Version 0.1a differs to v0.1 in a set of important aspects:

\begin{itemize}
\tightlist
\item
  v0.1 only simulates POC-l, DOC-l, POC-r, and DOC-r for water quality
  (meaning it essentially only captures the main metabolism reactions)
\item
  this means that in v0.1, GPP is a not dependent on any other state
  variable, but only on environmental conditions (light, nutrients)
\item
  v0.1 uses a time-dependent boundary condition for the internal
  availability of nutrients, i.e., TP
\item
  v0.1 does not include external mass fluxes, e.g., for POC-r and POC-l
\end{itemize}

\end{tcolorbox}

1D-AEMpy iteratively runs 8 modules to update the states of a set of
model variables.

\begin{figure}

{\centering \includegraphics{figures/conceptual_model.png}

}

\caption{\label{fig-best_model}Modularization of 1D-AEMpy.}

\end{figure}

\hypertarget{hydrodynamics}{%
\subsection{Hydrodynamics}\label{hydrodynamics}}

A one-dimensional hydrodynamic lake model was developed to simulate the
temperature, heat flux and stratification dynamics in a lake. The
algorithms are based on the eddy diffusion approach \textit{sensu}
(Henderson-Sellers (1985)) and the MyLake (Saloranta and Andersen
(2007)) model. Using the one-dimensional temperature diffusion equation
for heat transport, we neglected any inflows and outflows, mass losses
due to evaporation and water level changes: \[
    \frac{\partial h}{\partial t}=0
\]

\[
    A \frac{\partial T}{\partial t}=\frac{\partial}{\partial z}(A K_z \frac{\partial T}{\partial z}) + \frac{1}{{\rho_w c_p}}\frac{\partial H(z)}{\partial z}  + \frac{\partial A}{\partial z}\frac{H_{geo}}{\rho_w c_p}
\]

where \(h\) is the water level (m), \(A\) is lake area (m2), \(T\) is
water temperature (\textdegree C), \(t\) is time (s), \(K_z\) is the
vertical diffusion coefficient m2 s-1, \(H\) is internal heat generation
due to incoming solar radiation (W m2), \(\rho_w\) is water density (kg
m-3), \(c_p\) is specific heat content of water (J kg-1
\textdegree C-1), and \(H_{geo}\) is internal geothermal heat generation
(W m-2) (which was set to 0.1 W m-2 (Goudsmit et al. (2002))). Internal
heat generation is implemented based on Beer-Lambert law for attenuation
of short-wave radiation as a function of a constant light attenuation
coefficient: \[
    H(z)=\left(1-\alpha\right)I_s \text{exp}\left(-k_d z\right)
\] where \(\alpha\) is the albedo (\(-\)), \(I_s\) is total incident
short-wave radiation (W m-2), and \(k_d\) is a light attenuation
coefficient (m-1). For the boundary conditions, we assume a Neumann type
for the temperature diffusion equation at the atmosphere-surface
boundary, and a zero-flux Neumann type at the bottom: \[
    \rho_w c_p(K_z \frac{\partial T}{\partial z})_{surface}=H_{net}
\] \[
    K_z (\frac{\partial T}{\partial z})_{bottom}=0
\] where \(H_{net}\) is the net heat flux exchange between atmosphere
and water column (W m-2). The neat heat flux exchange consisted of four
terms: \[
    H_{net}=H_{lw}+H_{lwr}+H_v+H_c
\] where \(H_{lw}\) is the incoming long-wave radiation (W m-2),
\(H_{lwr}\) is emitted radiation from the water column (W m-2), \(H_v\)
is the latent heat flux (W m-2), and \(H_c\) is the sensible heat flux
(W m-2). Incoming and outgoing long-wave heat fluxes were derived using
the formulations from Livingstone and Imboden (1989) and Goudsmit et al.
(2002). The latent and sensible heat fluxes were calculated taking into
account atmospheric stability using the algorithm by Verburg and
Antenucci (2010).

The calculation of a temperature profile at every time step is
modularized into four steps: (a) heat generation from boundary
conditions, (b) ice and snow formation, (c) vertical diffusion, (d)
wind-induced mixing, and (e) convective overturn. The one-dimensional
temperature diffusion equation was discretized using the implicit
Crank-Nicolson scheme (Press et al. (2007)), which being second-order in
both space and time allows the modeling time step to be dynamic without
numerical stability issues. The model was implemented in Python 3.7 with
a default time step of \(\Delta t = 3,600\) s and a default spatial
discretization of \(\Delta z = 0.5\) m.

\hypertarget{a-heat-generation-from-boundary-conditions}{%
\subsubsection{a) Heat generation from boundary
conditions}\label{a-heat-generation-from-boundary-conditions}}

In the first step, the heat fluxes \(H\), \(H_{geo}\) and \(H_{net}\)
are applied over the vertical water column.

\hypertarget{b-ice-snow-and-snow-ice-formation}{%
\subsubsection{b) Ice, snow, and snow ice
formation}\label{b-ice-snow-and-snow-ice-formation}}

In the second step, the ice and snow cover algorithm from MyLake
@(saloranta\_mylakemulti-year\_2007) was applied to the model. Whenever
water temperatures were equal or below the freezing point of water (set
to 0 \textdegree C), ice formation was triggered. All layers with water
temperatures below the freezing point were set to 0 \textdegree C, and
the heat deficit from atmospheric heat exchange was converted into
latent heat of ice formation. Stefan's law was applied to calculate ice
thickness when air temperatures were below freezing point triggering ice
formation (e.g., Leppäranta (1993)): \[
     h_{ice}=\sqrt{h_{ice}^2+\frac{2 \kappa_{ice}}{\rho_{ice}L}(T_f-T_{ice})\Delta t}
\] where \(h_{ice}\) is ice thickness (\(m\)), \(\kappa_{ice}\) is
thermal conductivity of ice (\(W\) \(K^{-1}\) \(m^{-1}\)),
\(\rho_{ice}\) is ice density (\(kg\) \(m^{-3}\)), \(L\) is latent heat
of freezing (\(J\) \(kg^{-1}\)), \(T_f\) is water temperature at
freezing point (\(T_f =\) 0 \(^\circ C\)), and \(T_{ice}\) is the
temperature of the ice surface. The formation of a snow layer on top of
the ice layer depended on the amount of precipitation. Further, whenever
the weight of snow exceeded the buoyancy capacity of the ice layer,
enough water to offset the exceedance forms a snow ice layer with the
same properties as ice. When air temperatures were above the freezing
point, ice and snow growth ceased, and snow and ice melting were
initiated with ice melt requiring no snow to exist. Here, total energy
of melting was taken from the total heat flux \(H_{net}\). Once the ice
layer has disappeared, the default model routine continued. For more
details, we refer the reader to Saloranta and Andersen (2007).

\hypertarget{c-vertical-turbulent-diffusion}{%
\subsubsection{c) Vertical (turbulent)
diffusion}\label{c-vertical-turbulent-diffusion}}

In the third step, vertical turbulent diffusion between adjacent grid
cells was calculated. Here, we applied a centered difference
approximation for temperature at the next time step. The vertical
turbulent diffusion coefficients, \(K_t\), were calculated using an
empirical relationship depending on the Richardson number: \[
    K_z = \frac{K_0}{(1+5 Ri(z))^2} + K_m
\] where \(K_0\) is an adjustable parameter (set to \(10^{-2}\) m s-2)
\(R_i\) is the Richardson number, and \(K_m\) is the background eddy
diffusivity (Pacanowski and Philander (1981), Jabbari et al. (2023)).
The Richardson number was quantified as: \[
    R_i= \frac{-1 + [1+ 40 N^2 k^2 z^2 / \left({w^*}^2 \text{exp}(-2 k^* z)\right)]^{(1/2)}}{20}
\] with \(k\) as the Karman constant (\(k=0.4\)), and the squared
buoyancy frequency,
\(N^2=\frac{g}{\rho_w}\frac{\partial \rho_w}{\partial z}\) (\(s^{-2}\))
(Henderson-Sellers (1985)). Friction velocity \(w^*\) was calculated as:
\[
    w^*= C_D U_2 
\] where the drag coefficient \(C_D\) was set to \(1.3\) x \(10^{-3}\),
and \(U_2\) is the wind speed at 2 m above surface (m s-1). All values
of \(N^2\) less than \(7.0\) x \(10^{-5}\) s-2 were set to \(7.0\) x
\(10^{-5}\) s-2 (Hondzo and Stefan (1993)).

To replicate a lag in the mixing dynamics, we set the values of \(K_z\)
to the average between the current profile and the one from the previous
time step (Piccolroaz and Toffolon (2013)).

\hypertarget{numerical-implementation}{%
\paragraph{Numerical implementation}\label{numerical-implementation}}

The implicit Crnak-Nicolson scheme, whichis second-order derivative in
space and time, was applied to solve the one-dimensional temperature
transport equation for the diffusive transport. Here, we average the
response in space between the current and the next time step: \[
A \frac{\partial T}{\partial t}=A K_z \frac{\partial T}{\partial z}
\] \[
A \frac{T_i^{n+1} - T_i^n}{\Delta t} = \frac{1}{2} [A K_z \frac{T_{i+1}^{n+1} - 2 T_i^{n+1} + T_{i-1}^{n+1}}{\Delta z^2} +  A K_z \frac{T_{i+1}^{n} - 2 T_i^n + T_{i-1}^n}{\Delta z^2}]
\] Here, we can apply \(\alpha = A K_z \frac{\Delta t}{\Delta z^2}\),
transforming the equation to: \[
- \frac{\alpha}{2} T_{i+1}^{n+1} + (A+\alpha)T_{i}^{n+1}- \frac{\alpha}{2}T_{i-1}^{n+1} =\frac{\alpha}{2} T_{i+1}^{n} + (A-\alpha)T_{i}^{n}+ \frac{\alpha}{2}T_{i-1}^{n}
\] \[
- \frac{\alpha}{2} T_{i+1}^{n+1} + (A+\alpha)T_{i}^{n+1}- \frac{\alpha}{2}T_{i-1}^{n+1} = R_i^n
\] where the right-hand side quantity \(R_i^n\) is known at the
beginning of each time step.

In the current version, we asusme static Dirichlet boundary conditions
for the diffusive transport, which transforms the equation into a matrix
(given here for an example for 5 times 5 rows and columns): \[
\begin{bmatrix}
1 & 0 & 0 & 0 &0 \\
- \frac{\alpha}{2} & 1+\alpha & - \frac{\alpha}{2} & 0 & 0 \\
0 &- \frac{\alpha}{2} & 1+\alpha & - \frac{\alpha}{2} & 0  \\
0 & 0 &- \frac{\alpha}{2} & 1+\alpha & - \frac{\alpha}{2} \\
0 & 0 & 0 & 0 & 1
\end{bmatrix} 
\begin{bmatrix} T_1^{n+1} \\  
T_2^{n+1} \\ 
 T_3^{n+1} \\ 
  T_4^{n+1} \\ 
   T_5^{n+1} 
\end{bmatrix} =
\begin{bmatrix} T_1^n \\  
R_2^n \\ 
 R_3^n \\ 
  R_4^n \\ 
   T_5^n
\end{bmatrix} 
\]

\hypertarget{d-wind-induced-mixing}{%
\subsubsection{d) Wind-induced mixing}\label{d-wind-induced-mixing}}

To ensure an adequate representation of the effects of wind-induced
mixing, which was not incorporated into the turbulent diffusive
transport step, an additional step based on the concept of integral
energy was applied following the algorithms by Saloranta and Andersen
(2007) and Ford and Stefan (1980). Generally, the avaible external
turbulent kinetic energy (TKE) is compared to the potential energy of
the water column that is needed to lift up denser water from below a
mixed layer into a newly formed mixed layer until the external TKE is
depleted. TKE was quantified as: \[
TKE = C_{shelter} A_s \sqrt{\frac{\tau^3}{\rho_w}} \Delta t
\] where \(C_{shelter}\) is a wind-sheltering coefficient, and \(\tau\)
is the wind shear stress. \(C_{shelter}\) was parameterized based on
Hondzo and Stefan (1993): \[
C_{shelter} = 1.0 - exp(-0.3 A_s)
\] TKE is compared with the available potential energy (PE) in the water
column: \[
PE = g \Delta \rho_w \frac{V_{mixed} V_z}{V_{mixed} + V_z} (z_{mixed} +\Delta z_{M, z} - z_{M, mixed})
\] where \(\Delta \rho_w\) is the density difference between layer \(z\)
and the mixed-layer (epilimnion) density, \(V_{mixed}\) is the
mixed-layer volume, and \(\Delta z_{M, z}\) is the distance from layer
\(z\)'s center of mass, \(z_M\), to the bottom of the mixed-layer.

If TKE \(\geq\) PE, then layer \(z\) will be mixed into the mixed layer
(the epilimnion). This also includes water quality variables, which will
be volume-averaged. Once TKE \(<\) PE, the remaining energy will be used
for partial mixing as in Saloranta and Andersen (2007).

\hypertarget{e-convective-overturn}{%
\subsubsection{e) Convective overturn}\label{e-convective-overturn}}

In the final step, any density instabilities over the vertical water
column were mixed with the first stable layer below an unstable layer.
Here, we applied the area weighed mean of temperature between two layers
to calculate the new temperature of the previously unstable grid cell.
Density differences between two layers were averaged until the
difference was equal or less than \(1\) x \(10^{-3}\) kg m-3. \{An
arbitrary cutoff of \(1\) x \(10^{-3}\) kg m-3 was chosen to reduce
computational run time by preventing averaging of density profiles to
the fourth decimal point.\}

\hypertarget{water-quality}{%
\subsection{Water quality}\label{water-quality}}

The current implementation simulates dissolved oxygen, phytoplankton
biomass, a nutrient concentration (simulating reactive phosphorus),
labile particulate organic carbon (POC-l), labile dissolved organic
carbon (DOC-l), refractory particulate organic carbon (POC-r), and
refractory dissolved organic carbon (DOC-r) following a general equation
of: \[
A \frac{\partial C}{\partial t} + w \frac{\partial C}{\partial z} - \frac{\partial}{\partial z}(A K_z \frac{\partial C}{\partial z}) = P(C) - D(C)
\] where \(w\) is a sinking rate, \(P(C)\) is a production term, and
\(D(C)\) is a consumption (destruction) term.

The main feedback of water quality to hydrodynamics is via the light
extinction coefficient:

\[
k_d = k_{d, water} + k_{d, DOC} \overline{C_{DOC}} +  k_{d, POC} \overline{C_{POC}}
\] where there are specific light extinction coefficients for background
water, DOC and POC, \(k_{d, water}\), \(k_{d, DOC}\) and \(k_{d, POC}\),
respectively, as well as the average sum of DOC and POC concentrations,
\(\overline{C_{DOC}}\) and \(\overline{C_{POC}}\), respectively, from
the surface to the mean depth of the lake.

The water quality simulation is split into three modules dealing with
(f) effects of external and internal boundaries, (g) internal production
and conumption pathways, and (h) vertical transport.

\hypertarget{f-boundary-fluxes}{%
\subsubsection{f) Boundary fluxes}\label{f-boundary-fluxes}}

Boundary fluxes include mass addition from external loadings,
gas-exchange between the atmosphere and the water column, as well as
sediment consumption fluxes.

For mass fluxes, external loadings in m s-1 are added directly to the
layer adjacent to the atmosphere-water interface and are multiplied with
the time step.

Surface gas exchange for dissolved oxygen was parameterized as \[
F_{atm} = k_{p} (C_{DO, atm} - C_{DO}) A \Delta t
\] where \(k_{p}\) is the piston velocity, and \(C_{DO, atm}\) is the
saturated oxygen concentration. The piston velocity was calculated using
the empirical gas exchange model following Vachon and Prairie (2013)
based on lake area and wind velocity.

Sediment oxygen demand for the consumption of the dissolved oxygen
conentration \(C_{DO}\) (a sink for oxygen in the model) is quantified
as \[
SOD = (F_{DO}  + \frac{D_{diff}}{\delta} C_{DO} A ) \theta_{R}^{T-20} \Delta t
\] where \(F_{DO}\) is an idealized area flux rate (g m-2 s-1),
\(D_{diff}\) is the molecular oxygen diffusion coefficient (m2 s-1),
\(\delta\) is the thickness of the sediment diffusive boundary layer (m,
set to a constant value of 0.001 m ), and \(\theta\) is a temperature
multiplier, here for respiration \(R\). \(D_{diff}\) was quantified as a
function of water temperature following Han and Bartels (1996).

The release of nutrients from the sediment, \(SNR\) (a source of
nutrients in the model), was parameterized as a function depending on
dissolved oxygen availability (using Michaelis-Menten kinetics) and
temperature: \[
SNR = (F_{NUTR} \frac{C_{DO}}{k_{DO} + C_{DO}} A)\theta_{R}^{T-20} \Delta t
\] where \(F_{NUTR}\) is an idealized area flux rate (g m-2 s-1), and
\(k_{DO}\) is the half-saturation concentration of dissolved oxygen (g
m-3).

\begin{tcolorbox}[enhanced jigsaw, breakable, toprule=.15mm, rightrule=.15mm, bottomtitle=1mm, left=2mm, bottomrule=.15mm, leftrule=.75mm, opacityback=0, toptitle=1mm, arc=.35mm, coltitle=black, colframe=quarto-callout-note-color-frame, title=\textcolor{quarto-callout-note-color}{\faInfo}\hspace{0.5em}{Note}, opacitybacktitle=0.6, titlerule=0mm, colback=white, colbacktitle=quarto-callout-note-color!10!white]

In v0.1, SOD is quantified using Michaelis-Menten kinetics: \[
SOD = (F_{DO} \frac{C_{DO}}{k_{DO} + C_{DO}} A)\theta_{R}^{T-20} \Delta t
\]

Furter, GPP is quantified as a direct addition to oxygen, POC-l, and
DOC-l at every time step using a forward Euler scheme:

\[
GPP = R_{molar} (H p_1 I_P C_{nutr} \theta_{GPP}^{T-20})
\] where \(R_{molar}\) is the respective molar ratio to convert from
carbon units into something else, \(p_1\) is the conversion from
short-wave radiation to PAR (2.114), and \(I_P\) is a calibration
coeffcient.

\[
M_{DO}^{t} = M_{DO}^{t-1} + \Delta t * GPP 
\] \[
M_{POC-l}^{t} = M_{POC-l}^{t-1} + 0.2 \Delta t * GPP 
\] \[
M_{DOC-l}^{t} = M_{DOC-l}^{t-1} + 0.8 \Delta t * GPP 
\]

\end{tcolorbox}

\hypertarget{g-production-and-consumption}{%
\subsubsection{g) Production and
consumption}\label{g-production-and-consumption}}

The model tracks the mass dynamics of dissolved oxygen, \(M_{DO}\),
phytoplankton biomass, \(M_{phyto}\), nutrients, \(M_{nutr}\), labile
particulate organic carbon, \(M_{POC-l}\), labile dissolved organic
carbon, \(M_{DOC-l}\), refractory particulate organic carbon,
\(M_{POR-r}\), and refractory dissolved organic carbon, \(M_{POC-l}\).
The respective concentrations, \(C\) were quantified by dividing the
masses with the voume.

For production terms, gross primary production was quantified as \[
GPP = M_{phyto} R_{molar} (H p_1 I_P C_{nutr} \theta_{GPP}^{T-20})
\] where \(R_{molar}\) is the respective molar ratio to convert from
carbon units into something else, \(p_1\) is the conversion from
short-wave radiation to PAR (2.114), and \(I_P\) is a calibration
coeffcient.

Respiration was quantified as depending on oxygen availability \[
RSP = M_{i} R_{molar} R_i \frac{C_{DO}}{k_{DO} + C_{DO}} \theta_{R}^{T-20}
\] with \(M_{i}\) either being nutrient, POC-l, DOC-l, POC-r, or DOC-r,
and \(R_i\) is the respective respiration rate.

Phytoplankton growth was quantified similar to GPP \[
G = p_2 H p_1 \frac{C_{phyto}}{k_{phyto} + C_{phyto}} \theta_{GPP}^{T-20}
\] with \(p_2\) being the growth rate of algae (d-1).

The respective production and consumption terms for each variable are
listed below:

\[
P_{DO} - D_{DO} = GPP - (RSP_{POC-l} + RSP_{DOC-l} + RSP_{POC-r} + RSP_{DOC-r} + RSP_{nutr})
\] \[
P_{POC-l} - D_{POC-l} = 0.8 GPP - RSP_{POC-l} 
\] \[
P_{DOC-l} - D_{DOC-l} = (RSP_{POC-l} + 0.2 GPP) - RSP_{DOC-r} 
\] \[
P_{POC-r} - D_{POC-r} =  - RSP_{POC-r} 
\] \[
P_{DOC-r} - D_{DOC-r} = RSP_{POC-r} - RSP_{DOC-r} 
\]

\[
P_{phyto} - D_{phyto} = G M_{phyto} - p_3 M_{phyto}  \theta_{GPP}^{T-20}
\] with \(p_3\) as a grazing rate (d-1). \[
P_{nutr} - D_{nutr} = p_3 p_4 M_{phyto} R_{molar}  \theta_{GPP}^{T-20} - RSP_{nutr}
\] with \(p_4\) as the grazing ratio (hence, how much of the grazed
material is converted from the grazer into nutrients, similar to an
excretion rate).

\begin{tcolorbox}[enhanced jigsaw, breakable, toprule=.15mm, rightrule=.15mm, bottomtitle=1mm, left=2mm, bottomrule=.15mm, leftrule=.75mm, opacityback=0, toptitle=1mm, arc=.35mm, coltitle=black, colframe=quarto-callout-note-color-frame, title=\textcolor{quarto-callout-note-color}{\faInfo}\hspace{0.5em}{Note}, opacitybacktitle=0.6, titlerule=0mm, colback=white, colbacktitle=quarto-callout-note-color!10!white]

In v0.1, no production terms are used as GPP is included in the boundary
flux module. Therefore, the following consumption terms are used: \[
P_{DO} - D_{DO} = - (RSP_{POC-l} + RSP_{DOC-l} + RSP_{POC-r} + RSP_{DOC-r})
\] \[
P_{POC-l} - D_{POC-l} = - RSP_{POC-l} 
\] \[
P_{DOC-l} - D_{DOC-l} = - RSP_{DOC-r} 
\] \[
P_{POC-r} - D_{POC-r} =  - RSP_{POC-r} 
\] \[
P_{DOC-r} - D_{DOC-r} =  - RSP_{DOC-r} 
\]

\end{tcolorbox}

\hypertarget{numerical-implementation-1}{%
\paragraph{Numerical implementation}\label{numerical-implementation-1}}

For internal production and consumption fluxes, we applied the 2nd-order
Modified Patankar-Runge-Kutta scheme, which is mass conservative and
unconditionally positive (Burchard, Deleersnijder, and Meister (2003)):
\[
C_i^{(1)}=C_i^n + \Delta t(\sum_{j=1}^I P_{i,j}(C^n) - \sum_{j=1}^I D_{i,j}(C^n) \frac{C_i^{(1)}}{C_i^n})
\] \[
C_i^{n+1} = C_i^n + \frac{\Delta t}{2} (\sum_{j=1}^I (P_{i,j}(C^n) + p_{i,j}(C^{(1)})) \frac{C_j^{n+1}}{C_j^{(1)}} - \sum_{j=1}^I (D_{i,j}(C^n) + D_{i,j}(C^{(1)})) \frac{C_j^{n+1}}{C_j^{(1)}})
\] \[
\text{with } i = 1, ..., I
\] where \(i\) determines the current water quality constituent, \(P\)
is a source term, and \(D\) is a sink term.

\hypertarget{h-transport}{%
\subsubsection{h) Transport}\label{h-transport}}

For vertical transport, sinking and diffusion were parameterized for
particulate and dissolved substances, respectively. For phytoplankton
biomass, sinking and diffusion were both applied.

Sinking was conceptualized as a sinking loss that was applied to every
cell: \[
SL = M_i \frac{v_{i,settling}}{\Delta z} \Delta t
\] with \(v_{i,settling}\) as a constant settling velocity. Similarily,
a constant sedimentation rate, \(v_{sedimentation}\) was applied to the
grid cell adjacent to the sediment-water interface to remove mass from
the model over time.

Turbulent diffusion was applied to the dissolved substances according to
the previously described scheme in (c).

\hypertarget{model-parameterization}{%
\subsection{Model parameterization}\label{model-parameterization}}

\begin{longtable}[]{@{}
  >{\raggedright\arraybackslash}p{(\columnwidth - 6\tabcolsep) * \real{0.1667}}
  >{\raggedright\arraybackslash}p{(\columnwidth - 6\tabcolsep) * \real{0.2000}}
  >{\raggedleft\arraybackslash}p{(\columnwidth - 6\tabcolsep) * \real{0.4333}}
  >{\centering\arraybackslash}p{(\columnwidth - 6\tabcolsep) * \real{0.2000}}@{}}
\toprule\noalign{}
\begin{minipage}[b]{\linewidth}\raggedright
Model parameter
\end{minipage} & \begin{minipage}[b]{\linewidth}\raggedright
Symbol
\end{minipage} & \begin{minipage}[b]{\linewidth}\raggedleft
Description
\end{minipage} & \begin{minipage}[b]{\linewidth}\centering
Default value
\end{minipage} \\
\midrule\noalign{}
\endhead
\bottomrule\noalign{}
\endlastfoot
u & \(T\) & Initial temperature conditions (C) & - \\
o2 & \(M_{DO}\) & Initial oxygen conditions (g) & - \\
docr & \(M_{DOC-r}\) & Initial DOC-r conditions (g) & - \\
docl & \(M_{DOC-l}\) & Initial DOC-l conditions (g) & - \\
pocr & \(M_{POC-r}\) & Initial POC-r conditions (g) & - \\
pocl & \(M_{POC-l}\) & Initial POC-l conditions (g) & - \\
alg & \(M_{phyto}\) & Initial phytoplankton conditions (g) & - \\
nutr & \(M_{nutr}\) & Initial nutrient conditions (g) & - \\
startTime & - & Start time-date (s) & - \\
endTime & - & End time-date (s) & - \\
area & \(A\) & Depth-discrete areas (m2) & - \\
volume & \(V\) & Depth-discrete volumes (m3) & - \\
depth & \(z\) & Depth-discrete depths (m) & - \\
zmax & - & Maximum lake depth & - \\
nx & - & Maximum grid cell number & 50 \\
dt & \(\Delta t\) & Time step (s) & 3600 \\
dx & \(\Delta z\) & Space step (s) & 0.5 \\
daily\_meteo & - & Meteorological driver data & - \\
secview & - & Observed Secchi disk data (ignored since v0.1) & - \\
phosphorus\_data & - & Observed phosphorus data (ignored in v0.1a) &
- \\
ice & - & Initial ice condition, TRUE or FALSE & FALSE \\
Hi & \(h_{ice}\) & Initial ice thickness (m) & 0 \\
Hs & - & Initial snow thickness (m) & 0 \\
Hsi & - & Initial snow ice thickness (m) & 0 \\
iceT & - & Initial moving average temperature for ice (C) & 6 \\
supercooled & - & Initial amount of layers below freezing point & 0 \\
diffusion\_method & - & Numerical method for \(K_z\) &
``pacanowskiPhilander'' \\
scheme & - & Numerical method for diffusion & ``implicit'' \\
km & \(K_m\) & Background eddy diffusivity (m2s-1) & \(1.4 *10^{-7}\) \\
k0 & - & Adjustable eddy diffusivity parameter (m2s-1) &
\(1.0 *10^{-2}\) \\
weight\_kz & - & Weighting of eddy diffusivity & 0.5 \\
kd\_light & \(k_d\) & Light attenuation (m-1, ignored since v0.1) &
0.6 \\
densThresh & - & Density cutoff (kg m-3) & \(1.0 * 10^{-2}\) \\
albedo & \(\alpha\) & Albedo & 0.1 \\
eps & - & Emissivity of water & 0.97 \\
emissivity & - & Emissivity of water (same as eps) & 0.97 \\
sigma & - & Stefan-Boltzmann constant (W m-2K-4) & \(5.67 *10^{-8}\) \\
sw\_factor & - & Multiplier for short-wave radiation & 1.0 \\
wind\_factor & - & Multiplier for wind speed & 1.0 \\
at\_factor & - & Multiplier for air temperature & 1.0 \\
turb\_factor & - & Multiplier for turbulent heat fluxes & 1.0 \\
p2 & - & Ignored since v0.1 & 1.0 \\
B & - & Ignored since v0.1 & 0.61 \\
g & \(g\) & Gravitational acceleration (m s-2) & 9.81 \\
Cd & \(C_D\) & Momentum coefficient of wind & 0.0013 \\
meltP & - & Multiplier for ice melting & 1.0 \\
dt\_iceon\_avg & - & Ratio of ice forming temperature & 0.8 \\
Hgeo & \(H_{geo}\) & Geothermal heat flux (W m-2) & 0.1 \\
KEice & - & TKE multiplier for ice conditions (ignored since v0.1) &
0 \\
Ice\_min & - & Minimum ice thickness (m) & 0.1 \\
pgdl\_mode & - & Additional data processing & ``on'' \\
rho\_snow & \(\rho_{snow}\) & Initial snow density (kg m-3) & 250 \\
p\_max & - & Multiplier for GPP (ignored since v0.1) & 1/86400 \\
IP & \(I_P\) & Calibration coefficient for GPP &
\(3.0 *10^{-6}\)/86400 \\
theta\_npp & \(\theta_{GPP}\) & Arrhenius temperature multiplier &
1.0 \\
theta\_r & \(\theta_{R}\) & Arrhenius temperature multiplier & 1.08 \\
conversion\_constant & - & Multiplier for GPP (ignored since v0.1) &
\(1.0 *10^{-4}\) \\
sed\_sink & \(F_{nutr}\) & Idealized nutrient source rate (g m-2 s-1) &
0.01/86400 \\
k\_half & \(k_{DO}\) & Half-saturation concentration (g m-3) & 0.5 \\
resp\_docr & - & DOC-r respiration rate (s-1) & 0.008/86400 \\
resp\_docl & - & DOC-l respiration rate (s-1) & 0.008/86400 \\
resp\_pocr & - & POC-r respiration rate (s-1) & 0.08/86400 \\
resp\_pocl & - & POC-l respiration rate (s-1) & 0.08/86400 \\
grazing\_rate & \(p_3\) & Grazing rate of phytoplankton (s-1) &
\(3.0 *10^{-3}\)/86400 \\
pocr\_settling\_rate & \(v_{POC-r,settling}\) & POC-r settling rate (m
s-1) & 0.2/86400 \\
pocl\_settling\_rate & \(v_{POC-l,settling}\) & POC-l settling rate (m
s-1) & 0.2/86400 \\
algae\_settling\_rate & \(v_{phyto,settling}\) & Phytoplankton settling
rate (m s-1) & \(1.0 *10^{-5}\)/86400 \\
sediment\_rate & \(v_{sedimentation}\) & Sedimentation rate (m s-1) &
0.5/86400 \\
piston\_velocity & \(k_p\) & Piston velocity for gas exchange (m s-1,
ignored since v0.1) & 1.0/86400 \\
light\_water & \(k_{d, water}\) & Background light extinction (m-1) &
0.125 \\
light\_doc & \(k_{d, DOC}\) & DOC light extinction (m-1) & 0.02 \\
light\_poc & \(k_{d, POC}\) & POC light extinction (m-1) & 0.7 \\
mean\_depth & - & Mean depth of the lake (m) & \(\frac{\sum V}{A_s}\) \\
W\_str & \(C_{shelter}\) & Manual wind-sheltering coefficient &
``None'' \\
tp\_inflow & - & Boundary condition of total phosphorus (m-1, ignored in
v0.1a) & - \\
pocr\_inflow & - & Mass influx of POC-r (m-1, ignored in v0.1) & - \\
pocl\_inflow & - & Mass flux of POC-l (m-1, ignored in v0.1) & - \\
f\_sod & \(F_{DO}\) & Idealized oxygen sink rate (g m-2 s-1, ignored in
v0.1) & 0.1/86400 \\
d\_thick & \(\delta\) & Thickness of diffusive boundary layer (m) &
0.001 \\
growth\_rate & \(p_2\) & Phytoplankton growth ratio (s-1) &
\(1.0 *10^{-3}\)/86400 \\
grazing\_ratio & \(p_4\) & Grazing ratio & 0.1 \\
\end{longtable}

\hypertarget{refs}{}
\begin{CSLReferences}{1}{0}
\leavevmode\vadjust pre{\hypertarget{ref-burchard_patankar_2003}{}}%
Burchard, Hans, Eric Deleersnijder, and Andreas Meister. 2003. {``A
High-Order Conservative Patankar-Type Discretisation for Stiff Systems
of Production-Destruction Equations.''} \emph{Applied Numerical
Mathematics.} 47 (1): 1--30.
https://doi.org/\url{https://doi.org/10.1016/S0168-9274(03)00101-6}.

\leavevmode\vadjust pre{\hypertarget{ref-ford_thermal_1980}{}}%
Ford, Dennis E, and Heinz G Stefan. 1980. {``Thermal Predictions Using
Integral Energy Model.''} \emph{Journal of the Hydraulics Division} 106
(1): 39--55. \url{https://doi.org/10.1061/JYCEAJ.0005358}.

\leavevmode\vadjust pre{\hypertarget{ref-goudsmit_application_2002_epsilon}{}}%
Goudsmit, G.-H., H. Burchard, F. Peeters, and A. Wüest. 2002.
{``Application of {k-\textepsilon} Turbulence Models to Enclosed Basins:
{The} Role of Internal Seiches.''} \emph{Journal of Geophysical
Research: Oceans} 107 (C12): 23-1-23-13.
\url{https://doi.org/10.1029/2001JC000954}.

\leavevmode\vadjust pre{\hypertarget{ref-han_oxygen_1996}{}}%
Han, Ping, and David M. Bartels. 1996. {``Temperature Dependence of
Oxygen Diffusion in H2O and D2O.''} \emph{J. Phys. Chem.} 100 (13):
5597--5602. https://doi.org/\url{https://doi.org/10.1021/jp952903y}.

\leavevmode\vadjust pre{\hypertarget{ref-henderson-sellers_new_1985}{}}%
Henderson-Sellers, B. 1985. {``New Formulation of Eddy Diffusion
Thermocline Models.''} \emph{Applied Mathematical Modelling} 9 (6):
441--46. \url{https://doi.org/10.1016/0307-904X(85)90110-6}.

\leavevmode\vadjust pre{\hypertarget{ref-hondzo_lake_1993}{}}%
Hondzo, Midhat, and Heinz G. Stefan. 1993. {``Lake {Water} {Temperature}
{Simulation} {Model}.''} \emph{Journal of Hydraulic Engineering} 119
(11): 1251--73.
\url{https://doi.org/10.1061/(ASCE)0733-9429(1993)119:11(1251)}.

\leavevmode\vadjust pre{\hypertarget{ref-jabbari_nearshore_2023}{}}%
Jabbari, Aidin, Reza Valipur, Josef D. Ackerman, and Yerubandi R. Rao.
2023. {``{Nearshore-offshore exchanges by enhanced turbulent mixing
along the north shore of Lake Ontario}.''} \emph{Journal of Great Lakes
Research} 49 (3): 596--607.
https://doi.org/\url{https://doi.org/10.1016/j.jglr.2023.03.010}.

\leavevmode\vadjust pre{\hypertarget{ref-lepparanta_review_1993}{}}%
Leppäranta, Matti. 1993. {``A Review of Analytical Models of Sea‐ice
Growth.''} \emph{Atmosphere-Ocean} 31 (1): 123--38.
\url{https://doi.org/10.1080/07055900.1993.9649465}.

\leavevmode\vadjust pre{\hypertarget{ref-livingstone_annual_1989}{}}%
Livingstone, David M., and Dieter M. Imboden. 1989. {``Annual Heat
Balance and Equilibrium Temperature of {Lake} {Aegeri},
{Switzerland}.''} \emph{Aquatic Sciences} 51 (4): 351--69.
\url{https://doi.org/10.1007/BF00877177}.

\leavevmode\vadjust pre{\hypertarget{ref-pacanowski_paramterization_1981}{}}%
Pacanowski, R. C., and S. G. H. Philander. 1981. {``{Parameterization of
Vertical Mixing in Numerical Models of Tropical Oceans}.''}
\emph{Journal of Physical Oceanography} 1: 1443--51.

\leavevmode\vadjust pre{\hypertarget{ref-piccolroaz_deep_2013}{}}%
Piccolroaz, Sebastiano, and Marco Toffolon. 2013. {``Deep Water Renewal
in {Lake} {Baikal}: {A} Model for Long-Term Analyses: {Deep} {Water}
{Renewal} in {Lake} {Baikal}.''} \emph{Journal of Geophysical Research:
Oceans} 118 (12): 6717--33. \url{https://doi.org/10.1002/2013JC009029}.

\leavevmode\vadjust pre{\hypertarget{ref-press_numericalrecipes}{}}%
Press, W. H., S. A. Teukolsky, W. T. Vetterling, and B. P. Flannery.
2007. \emph{{Numerical} {Recipes:} {The} {Art} of {Scientific}
{Computing}}. Cambridge University Press.

\leavevmode\vadjust pre{\hypertarget{ref-saloranta_mylakemulti-year_2007}{}}%
Saloranta, Tuomo M., and Tom Andersen. 2007. {``{MyLake}---{A}
Multi-Year Lake Simulation Model Code Suitable for Uncertainty and
Sensitivity Analysis Simulations.''} \emph{Ecological Modelling},
Uncertainty in {Ecological} {Models}, 207 (1): 45--60.
\url{https://doi.org/10.1016/j.ecolmodel.2007.03.018}.

\leavevmode\vadjust pre{\hypertarget{ref-vachon_ecosystem_2013}{}}%
Vachon, Dominic, and Yves T. Prairie. 2013. {``The Ecosystem Size and
Shape Dependence of Gas Transfer Velocity Versus Wind Speed
Relationships in Lakes.''} Edited by Ralph Smith. \emph{Canadian Journal
of Fisheries and Aquatic Sciences} 70 (12): 1757--64.
\url{https://doi.org/10.1139/cjfas-2013-0241}.

\leavevmode\vadjust pre{\hypertarget{ref-verburg_persistent_2010}{}}%
Verburg, Piet, and Jason P. Antenucci. 2010. {``Persistent Unstable
Atmospheric Boundary Layer Enhances Sensible and Latent Heat Loss in a
Tropical Great Lake: {Lake} {Tanganyika}.''} \emph{Journal of
Geophysical Research} 115 (D11): D11109.
\url{https://doi.org/10.1029/2009JD012839}.

\end{CSLReferences}



\end{document}
